\documentclass{article}
\usepackage{geometry}
\geometry{a4paper, margin=1in}
\usepackage[table]{xcolor}
\usepackage{graphicx}
\usepackage{float}
\usepackage{enumitem}
\usepackage{fancyhdr}

\pagestyle{fancy}
\fancyhf{}
\rfoot{\thepage}
\begin{document}

\begin{center}
    \section*{Background Study of the Domain}
\end{center}

The Indian education system is known for its rigorous curriculum and high academic standards. The system relies heavily on entrance exams to determine which students are admitted into top institutions. As a result, the entrance exam market has become a critical part of the Indian education system, with millions of students competing each year for a limited number of seats. The Indian entrance exam market is a complex and dynamic space that has significant implications for the country's education system and economy.

The market is highly competitive, with multiple players vying for a share of the market. These players include government bodies, private companies, and coaching centers, each with their own strengths and weaknesses. Despite the market's importance, there is a lack of comprehensive research on the Indian entrance exam market. While there are studies on specific aspects of the market, such as the impact of coaching centers on student performance, there is a need for a comprehensive overview of the market's competitive landscape. The lack of research is concerning, given the market's size and importance. The entrance exam market in India is estimated to be worth billions of dollars, and it has significant implications for the country's education system and economy. The market is also prone to corruption and malpractice, which can have far-reaching consequences for students and institutions. The need for a comprehensive study of the Indian entrance exam market is underscored by the fact that entrance exams have a significant impact on students' academic and career trajectories.

The results of these exams determine which students are admitted into top institutions, which in turn can have a significant impact on their future prospects. The Indian government has introduced several reforms in recent years, aimed at improving the quality and accessibility of education. These reforms include changes to the entrance exam system, such as the introduction of online exams and the reduction of exam fees. Given these changes, there is a need for a comprehensive study of the Indian entrance exam market that takes into account the evolving education landscape.

The study of the Indian entrance exam market is also important in the context of the COVID-19 pandemic. The pandemic has had a significant impact on the education sector, with schools and colleges closed for extended periods. This has led to a rise in the use of online education and the adoption of new technologies. The pandemic has also had an impact on the entrance exam market, with several exams being postponed or canceled. This has created uncertainty and anxiety among students, and has highlighted the need for a more resilient and adaptable entrance exam system. Lastly, the Indian entrance exam market is a critical component of the country's education system and economy. Despite its importance, there is a lack of comprehensive research on the market's competitive landscape. The need for a comprehensive study of the market is underscored by the changing education landscape, the impact of the COVID-19 pandemic, and the market's significant implications for students and institutions. A comprehensive study of the Indian entrance exam market can provide valuable insights into the market's strengths and weaknesses, and inform policy interventions and business strategies that could improve the market's transparency, fairness, and accessibility.
\vspace{1cm}
\section*{Emerging Trends and Challenges}

\subsection*{Online Learning}

With the rise of technology, online learning has emerged as a major trend in the entrance exam market. Students are increasingly turning to online resources to prepare for their exams, as it offers them flexibility and access to study material anytime, anywhere. This has led to a growth in the online coaching and study material market, and several players have entered the market to meet the demand.

\subsection*{Personalized Learning}

Another emerging trend in the entrance exam market is personalized learning. With advances in artificial intelligence and machine learning, companies are now able to offer personalized coaching and study material to students, based on their individual learning styles and strengths. This trend is expected to continue to grow, as companies look for new ways to differentiate themselves and meet the needs of students.

\subsection*{Rising Costs}

The cost of entrance exam coaching and study material has been on the rise in recent years, posing a challenge for students from lower-income backgrounds. This has led to a growing demand for affordable coaching services and study material, and several players have entered the market to meet this demand.

\subsection*{Increasing Competition}

The entrance exam market in India is highly competitive, with several players vying for a share of the market. This has led to a rise in the quality of coaching services and study material, as companies strive to differentiate themselves and attract students. However, this competition has also led to a rise in unethical practices, such as cheating and fraud, posing a challenge for students and the industry as a whole.

\subsection*{Changing Exam Formats}

Entrance exams in India are constantly evolving, with changes in exam formats and syllabi. This has led to a need for updated coaching services and study material, and companies must adapt to these changes to remain relevant and meet the needs of students.

\subsection*{Mental Health}

The pressure to perform well in entrance exams can take a toll on students' mental health, leading to stress and anxiety. This has led to a growing demand for coaching services and study material that focus on mental health and well-being, and companies are beginning to respond to this demand.

\subsection*{Inequality}

The entrance exam market in India is also affected by issues of inequality, with students from lower-income backgrounds often having less access to coaching services and study material. This poses a challenge for the industry, as it must find ways to ensure that all students have equal access to resources and opportunities.

\section*{Personalized Learning}

\textit{“Everybody is a genius. But if you judge a fish by its ability to climb a tree, it will live its whole life believing that it is stupid.”}

Personalized learning, also known as individualized learning, is an educational approach that aims to tailor instruction, content, pace, and learning activities to meet the unique needs, interests, and abilities of each student. Rather than employing a one-size-fits-all approach to education, personalized learning recognizes that students have different learning styles, preferences, strengths, and areas for improvement.

\textbf{Key features of personalized learning include:}

\begin{itemize}[label=$\bullet$]
\item \textbf{Customized Learning Paths:} Students are provided with learning experiences that are tailored to their specific needs and learning goals. This may involve assessing students' prior knowledge and skills, identifying areas of strength and weakness, and designing individualized learning plans accordingly.
\item \textbf{Differentiated Instruction:} Instructional strategies are adapted to accommodate diverse learning styles and preferences. Teachers may employ a variety of teaching methods, resources, and technologies to engage students and address their individual learning needs.
\item \textbf{Flexible Pacing:} Students have the flexibility to progress through learning materials at their own pace. This allows students to spend more time on concepts they find challenging and move more quickly through topics they already understand.
\item \textbf{Personalized Content:} Learning materials, resources, and assignments are selected or created to align with students' interests, backgrounds, and abilities. This may involve incorporating multimedia, real-world examples, and culturally relevant content to enhance relevance and engagement.
\end{itemize}

\begin{center}
    \section*{Existing Solutions}
\end{center}

\subsection*{BYJU'S}

\textbf{Strengths:}
\begin{itemize}
\item Adaptive Learning Paths: BYJU'S utilises adaptive learning technology to customise the learning journey for each student. This means that the platform dynamically adjusts the content and difficulty level based on the individual student's progress and understanding.
\item Engaging Multimedia Content: BYJU'S offers engaging and interactive multimedia content, including high-quality videos, animations, simulations, and interactive quizzes.
\item Real-Time Feedback and Analytics: BYJU'S offers real-time feedback and performance analytics to students, parents, and teachers.
\end{itemize}

\textbf{Gaps:}
\begin{itemize}
\item Limited Content Diversity: BYJU'S may struggle to provide a wide range of content diversity. While it offers comprehensive study materials and resources for various subjects and topics, there might be limitations in the breadth and depth of content available. This lack of diversity could hinder the platform's ability to cater to the varied learning needs and preferences of students, especially those who benefit from exposure to a diverse range of educational materials.
\item Potential Lack of Personalized Recommendations: While BYJU'S often employs adaptive learning paths that adjust content based on individual student progress and understanding, there might be gaps in providing personalized recommendations. The platform may not offer a robust system for tailoring content suggestions and resources to meet the specific learning needs, interests, and preferences of each student.
\end{itemize}

\subsection*{Khan Academy}

\textbf{Strengths:}
\begin{itemize}
\item Provides Personalized Learning Paths: Khan Academy offers personalised learning paths for students, allowing them to progress through content at their own pace. The platform adapts to individual learning needs, providing recommendations and resources based on students' performance and mastery levels.
\item Progress Tracking: Students have access to a dashboard that enables them to track their progress effectively. They can monitor their completion of activities, view the time spent on each topic, and assess their proficiency levels. This feature helps students identify areas where they excel and areas that may require further attention.
\end{itemize}


\textbf{Gaps:}
\begin{itemize}
    \item Limited Real-Time Adaptability: While Khan Academy provides personalised learning paths, its adaptability in real-time may be limited. The platform may not dynamically adjust content or difficulty levels based on immediate student interactions or performance. This lack of real-time adaptability could result in less immediate support or challenges for students as they progress through the material.
    \item Personalised Recommendations: While Khan Academy offers personalised learning paths, its range of personalised recommendations may be somewhat limited. The platform primarily focuses on video content, and while it provides related exercises and resources, it may not offer a wide variety of personalised recommendations across different learning modalities. This limitation could hinder students who benefit from diverse learning experiences or who require additional support in specific areas.
    \item Focuses Mainly on Video Content: Khan Academy's primary mode of instruction is through video content. While videos can be an effective learning tool for many students, some learners may find them less engaging or may prefer alternative formats such as interactive exercises, simulations, or hands-on activities. The reliance on video content may limit the platform's ability to cater to diverse learning preferences and needs.
\end{itemize}

\subsubsection*{ThinkMerit.in}

\textbf{Strengths:}
\begin{itemize}
    \item Thinkmerit.in offers a feature called "Customised Test," enabling students to create their own tests by selecting chapters from different subjects. This feature provides flexibility and customization, allowing students to focus on specific topics or chapters they want to practise.
    \item The platform provides a Course Planner tool that plans the course date-wise using algorithms. This tool assigns weightage and priority to important chapters, helping students organise their study schedule effectively.
\end{itemize}

\textbf{Gaps:}
\begin{itemize}
    \item While the Course Planner in Thinkmerit.in utilises algorithms to plan the course date-wise and prioritise chapters, it lacks the capability to dynamically update or adjust the course based on students' performance. 
    \item Another potential gap in Thinkmerit.in is the absence of an interactive feedback mechanism that allows students to provide input on their learning experience, course preferences, and areas of difficulty.
\end{itemize}

\subsubsection*{Knockout JEE Main program by Careers360}

\textbf{Strengths:}
\begin{itemize}
    \item Topic-wise Study Material: The Knockout JEE Main program by Careers360 offers "Smart Study Material," indicating that it provides study materials organised by topic. This organization allows students to focus on specific subjects or concepts, enabling targeted learning and revision.
    \item Adaptive Time Table: The program features an "Adaptive Time Table," suggesting that it offers personalised learning schedules tailored to individual students' needs and preferences. Personalised timetables take into account factors such as students' strengths, weaknesses, study pace, and available study hours. By providing customised study plans, the program assists students in effectively managing their time and staying on track with their preparation.
\end{itemize}

\textbf{Gaps:}
\begin{itemize}
    \item Clarity on Practice Questions: The Knockout JEE Main program lacks clear details regarding the number and diversity of practice questions included in the study materials. While it mentions "Smart Study Material," it does not provide specific information about the quantity or variety of practice questions available for each topic.
    \item Absence of Recommendation System: The program does not incorporate a recommendation system to suggest additional resources, practice materials, or study strategies based on students' individual learning needs and performance. A recommendation system could analyse students' study patterns, performance metrics, and areas of weakness to provide personalised recommendations for supplementary resources, targeted practice exercises, or specific study techniques.
\end{itemize}

\section*{Conclusion}

In conclusion, the entrance exam market in India is undergoing significant changes driven by technological advancements, changing student preferences, and evolving educational policies. Online learning and personalised learning are emerging as dominant trends, offering students greater flexibility, accessibility, and customisation in their exam preparation journey. However, challenges such as rising costs, increasing competition, and issues of inequality persist, highlighting the need for innovative solutions and policy interventions. Existing solutions such as BYJU'S, Khan Academy, ThinkMerit.in, and the Knockout JEE Main program by Careers360 offer promising features but also have gaps that need to be addressed to provide a more comprehensive and effective learning experience for students.

Moving forward, it is essential for stakeholders in the entrance exam market, including educational institutions, policymakers, and edtech companies, to collaborate and innovate to address the evolving needs and challenges of students. This may involve leveraging emerging technologies such as artificial intelligence and machine learning to develop more adaptive and personalised learning solutions, as well as implementing policies to ensure equitable access to education resources and opportunities.

By fostering innovation, promoting inclusivity, and prioritising student well-being, stakeholders can contribute to building a more robust and resilient entrance exam ecosystem that empowers all students to achieve their full potential. Through continuous research, evaluation, and improvement, the entrance exam market in India can evolve to better serve the needs of students and contribute to the country's overall educational excellence and societal development.

In conclusion, while the entrance exam market in India faces various challenges, it also presents numerous opportunities for positive change and innovation. By addressing the emerging trends and challenges discussed in this report and leveraging existing and emerging solutions, stakeholders can work together to create a more equitable, accessible, and effective entrance exam ecosystem that benefits all students.
\newpage
\begin{center}
    \section*{Multiple Solution}
\end{center}

\begin{enumerate}
    \item \textbf{Recommendation on the basis of content base without ML}
    \begin{itemize}
        \item Track and record the user’s activity on the website and store his choices.
        \item A database of the questions will be hardcoded.
        \item Assign points to each question, topic and subject corresponding to each student.
        \item Update flags and user progress according to user activity, and recommend new questions as per the updated progress.
        \item For a new student some Pre-fixed questions will be shown. 
        \item Student will be recommended for the topics, and questions in which his performance is weak.
    \end{itemize}

    \begin{figure}[htbp]
    \centering
    \includegraphics[width=1.0\textwidth]{Screenshot (743).png} 
    
    \label{fig:Flow}
    \end{figure}

    \item \textbf{Recommendation on the basis of content base with ML}
    \item \textbf{Recommendation on the basis of collaborative filtering}
    \item \textbf{Recommendation based on some technologies}
    \begin{itemize}
        \item LSTM
        
        \item GRU
        
        \item MATRIX FACTORIZATION
        
    \end{itemize}
    

\end{enumerate}
\newpage



\newpage
\begin{center}
    \section*{Proposed Solution}
\end{center}
\section*{Effectiveness Metrics:}
\begin{enumerate}[label=\alph*)]
    \item Click-Through Rate (CTR): Measures the proportion of recommendations that are clicked on or interacted with by users. A higher CTR indicates that recommendations are relevant and engaging for users.
    \item Conversion Rate: Measures the proportion of recommended resources that lead to a desired outcome, such as completing a learning module, solving a problem, or mastering a concept.
    \item User Engagement Metrics: Includes metrics such as time spent on recommended resources, frequency of interactions, and depth of engagement. Higher levels of user engagement indicate that recommendations are effectively capturing and maintaining students' interest and attention.
    \item Learning Outcome Improvement: Evaluates the impact of recommendations on students' learning outcomes, such as improvements in test scores, knowledge retention, or performance in assessments. This could be assessed through pre- and post-recommendation assessments or longitudinal studies.
\end{enumerate}

\section*{Model Architecture:}
\textbf{Technique Using:} LSTM or GRU:
\begin{itemize}
    \item Choosing the RNN architecture (LSTM or GRU) that yields better results for the task.
\end{itemize}

\textbf{Layers in Architecture:}
\begin{itemize}
    \item Embedding Layer: Converts categorical variables like question IDs or topics into dense vectors for efficient processing.
    \item Attention Mechanism: Dynamically weighs the importance of different features, such as past questions or topics, in the student's interaction history.
    \item LSTM Layer: Captures long-term dependencies in sequential data, such as the student's interaction history with questions.
    \item Dense Layer: Learns non-linear relationships between features extracted by previous layers and makes predictions about the next question to recommend.
\end{itemize}

\textbf{Input Format:}
\begin{itemize}
    \item Topic Sequence level: A sequence of topics attempted by the student.
    \item Question Sequence with each info: A sequence of questions attempted by the student. (time\_spent, accuracy, question points which is used in solve table)
\end{itemize}

\textbf{Topic Dependency Graph:} 
\begin{itemize}
    \item Encode the topic dependency graph with weights into a suitable format, such as an adjacency matrix or an edge list.
    \item Include the encoded graph as an additional input to the model.
\end{itemize}

\textbf{Topic Weights:} Weighted importance of each topic based on its relevance (dependency on another topic) and frequency in JEE exam.

\textbf{Output Format:}
\begin{itemize}
    \item Next Question Prediction: The model predicts the next question that the student should attempt based on the input sequences and auxiliary inputs.
    \item Probability Distribution: The model outputs a probability distribution over all possible questions, indicating the likelihood of each question being recommended next.
\end{itemize}

\section*{Schema of Student relevant in recommendation:}
\begin{itemize}
    \item Level (in student table)
    \item Accuracy for each question (in solve table)
    \item Time spent in each question. (in solve table)
    \item Avg time spent on all questions for each user (derived from solve table in student table)
    \item Topic array of each user. (in student table topic\_array\_level)
\end{itemize}

\section*{Schema of Question table relevant in Adaptation and recommendation:}
\begin{itemize}
    \item Acceptance rate
    \item Exam Frequency
    \item Difficulty
    \item Points
    \item Avg time spent on each question for all users attempted. (derived from solve table in question table as avg\_timespent\_all\_users)
    \item Topic of question
\end{itemize}

\textbf{RNN algorithms for sequential Recommendation:}
\begin{itemize}
    \item LSTM (Long Short term memory RNN)
    \item GRU (Gated Recurrent Network)
    \item GNN (Graph Neural Network)
\end{itemize}

\begin{figure}[H]
    \centering
    \includegraphics[width=0.5\textwidth]{Topic_Dependency_Graph-Page-2.drawio (1).png}
    \caption{Flow Diagram for Recommendation}
    \label{fig:example}
\end{figure}


\textbf{Flow:}

\textbf{New User:}
\begin{enumerate}
    \item Survey:
    \begin{enumerate}
        \item Asking about his level:
        \begin{enumerate}
            \item Naive User
            \item Intermediate User
            \item Experienced User
        \end{enumerate}
        \item Asking topics he/she knew as per the level he/she selected:
        \begin{enumerate}
            \item Asking users to fill check boxes.
        \end{enumerate}
        \item Taking a test based on his/her level and topic he/she selected. (for verifying the credibility of information)
        \item Changing the user’s level as per the test and info provided, i.e., adding topics as his/her strengths and weaknesses.
    \end{enumerate}
    \item Existing User:
    \begin{enumerate}
        \item For existing user, we will predict questions based on previous session analysis, where the data provided at each timestamp is:
        \begin{itemize}
            \item Topic Sequence level: A sequence of topics attempted by the student.
            \item Question Sequence with each info: A sequence of questions attempted by the student. (time\_spent, accuracy, question points which is used in solve table)
            \item Topic Dependency Graph:
            \begin{itemize}
                \item Encode the topic dependency graph with weights into a suitable format, such as an adjacency matrix or an edge list.
                \item Include the encoded graph by finding node embeddings.
            \end{itemize}
        \end{itemize}
    \end{enumerate}
    \item Topic Weights: Weighted importance of each topic based on its relevance (dependency on another topic) and frequency in JEE exam. Based on this data, our model will predict questions.
\end{enumerate}

\textbf{Model Steps:}
\begin{itemize}
    \item Input User Interaction Data:
    \begin{itemize}
        \item Collect data on the user's past interactions, including topic sequences and question sequences with associated information (time spent, accuracy, question points).
    \end{itemize}
    \item Data Preprocessing:
    \begin{itemize}
        \item Encode topics and questions as numerical or categorical representations.
        \item Organise the data into sequences representing each user's interaction history.
    \end{itemize}
    \item Build Graph:
    \begin{itemize}
        \item Encode the topic dependency graph with weights into a suitable format, such as an adjacency matrix or an edge list.
        \item Include the encoded graph as an additional input to the model.
    \end{itemize}
    \item Graph Embedding:
    \begin{itemize}
        \item Embed the graph of topics into a continuous vector space using techniques such as Graph Neural Networks or node embedding algorithms.
    \end{itemize}
    \item Model Input:
    \begin{itemize}
        \item Input the preprocessed data, including topic sequences, question sequences, and graph embeddings, into the model.
    \end{itemize}
    \item Model Prediction:
    \begin{itemize}
        \item Utilise the LSTM or GRU model architecture to predict the next question that the student should attempt.
    \end{itemize}
    \item Post-processing:
    \begin{itemize}
        \item Optionally, perform any additional processing or filtering of the model output.
    \end{itemize}
    \item Provide Prediction:
    \begin{itemize}
        \item Present the predicted question to the student for further engagement.
    \end{itemize}
\end{itemize}

\section*{\textbf{Dependency Graph :}}
\begin{figure}[H]
    \centering
    \includegraphics[width=1\textwidth]{Topic_Dependency_Graph-Page-1 (1).jpg}
    \caption{Dummy Dependency Graph for Physics}
    \label{fig:example}
\end{figure}

\section*{\textbf{Cold Start Problem:}}
\begin{enumerate}
    \item Address the cold start problem by incorporating techniques such as:
    \item Conducting tests and changing the topic graph of students correspondingly.
    \item To get difficulty points of questions, we'll conduct tests where we will assign difficulty points based on the basis of their average acceptance rate on all tests.
\end{enumerate}

\section*{\textbf{Design:}}

\subsection*{ER Diagram:}

\begin{figure}[H]
    \centering
    \includegraphics[width=0.5\textwidth]{ER_JEECODE.drawio.png}
    \caption{ER Diagram }
    \label{fig:example}
\end{figure}

\subsection*{Sequence Diagram:}

\begin{figure}[H]
    \centering
    \includegraphics[width=0.5\textwidth]{SequenceDiagram.drawio (1).png}
    \caption{Sequence Diagram }
    \label{fig:example}
\end{figure}

\subsection*{Activity Diagram:}
\subsubsection*{Student Attempting Question}

\begin{figure}[H]
    \centering
    \includegraphics[width=0.5\textwidth]{Activity-Page_QuestionAttempted.drawio.png}
    \caption{Student Attempting Question}
    \label{fig:example}
\end{figure}


\begin{figure}[H]
    \centering
    \includegraphics[width=0.5\textwidth]{Activity-Page-Q2.drawio.png}
    \caption{Student Attempting Question}
    \label{fig:example}
\end{figure}


\subsubsection*{Student Accessing Discussion Forum}

\begin{figure}[H]
    \centering
    \includegraphics[width=0.5\textwidth]{Activity-Page_discussion.drawio.png}
    \caption{Student Accessing Discussion Forum}
    \label{fig:example}
\end{figure}

\subsection*{Class Diagram:}

\begin{figure}[H]
    \centering
    \includegraphics[width=0.5\textwidth]{ClassDiagram_JeeCode.drawio.png}
    \caption{Class Diagram}
    \label{fig:example}
\end{figure}

\subsection*{Use Case Diagram:}

\begin{figure}[H]
    \centering
    \includegraphics[width=0.5\textwidth]{UseCase.drawio.png}
    \caption{Use Case Diagram}
    \label{fig:example}
\end{figure}

\newpage
\begin{center}
    \section*{Analysis}
\end{center}

\subsection*{Domain Understanding}
\begin{itemize}
    \item \textbf{Recommendation:} Our Model will recommend topics and questions to the user based on his previous actions, level, and interaction with model.
    \item \textbf{Adaption:} Updates in the database (Question difficulty and level) and weights (Dependency graph) will be committed as per the feedbacks (acceptance rate) and insertions.
    \item \textbf{Personalization:} Personalized recommendation based on previous activities i.e. prerequisites satisfaction, and by user survey (Coldstart - New user).
    \item \textbf{Student as user:} As per the major project domain, we are primarily focusing only on the JEE students as the stakeholders.
\end{itemize}

\subsection*{Data Exploration}
\begin{itemize}
    \item \textbf{Collection:} Data will be primarily collected from a JEE coaching Institute, and will be modified with the help of subject experts. Imp. attributes for recommendation:
    \begin{itemize}
        \item Student: Level, Standard, AttemptQuestions, AvgTime, AvgTime\_Subject
        
        \item Question: Difficulty, Points, Acceptance rate, topic, TotalAttempts, Time
        
        \item Topic: Question Array, Prerequisites, Difficulty rate, Dependency\_position
    \end{itemize}
    \item \textbf{Preprocessing:} Initially topic difficulty level, difficulty (easy..), and Null values will be handled with the help of experts. Topic interdependency will be drawn (Dependency graph).
    \item \textbf{Data:} Analysis Majorly effective when we will have processed data by the user, Current data is not processed, Can't identify trends.
\end{itemize}

\subsection*{Behaviour Analysis}
\begin{itemize}
    \item \textbf{User analysis:}
    \begin{itemize}
        \item Demographics: Age, gender, geo\_location, education\_background
        \item Learning Pref.: LearningStyle (visual, auditory..), contentDelivery (Video, quiz, text)
        \item Engagement Patterns: Frequency and duration of user interaction, session length
        \item Performance History: Scores, Completion rates, all progress tracking items......
        \item Interests and pref.: UI, subject and topic matter, solutions of questions.
    \end{itemize}
    \item \textbf{Item analysis:}
    \begin{itemize}
        \item Question Complexity: Diff\_level, bloom taxonomy, application
        \item Question type: Single correct, multiple, T/F, Number, text 
        \item Item Importance: Popularity, freq, Imp, success rates, views
        \item Relevance: Relevant topic, subject, curriculum of exam
    \end{itemize}
    \item \textbf{Interaction Question:} response time, Correctness of Responses, Analysis Engagement Metrics, Progress Tracking, Feedback and Support
\end{itemize}

\section*{Accuracy Metrics}
\begin{itemize}
    \item \textbf{Precision:} The proportion of relevant learning resources or questions among the top recommendations provided to students.
    \item \textbf{Recall:} Measures the proportion of relevant recommendations retrieved out of all the items in the dataset. It indicates the system's ability to recommend all relevant items to users within the top recommendations.
    \item \textbf{Mean Average Precision (MAP):} How well a recommendation system ranks items for a given user compared to the ground truth.
    \item \textbf{Normalized Discounted Cumulative Gain (NDCG):} Evaluates the ranking quality of recommended items by considering both relevance and position in the list of recommendations. Higher-ranked relevant items contribute more to the score.
\end{itemize}


\end{document}



